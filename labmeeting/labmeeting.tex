\documentclass{beamer}
\mode<presentation>
{
	\usetheme{Malmoe}
%	\usetheme{Darmstadt}
}

\usepackage[absolute,overlay]{textpos}

\newcommand*\oldmacro{}%
\let\oldmacro\insertshorttitle%
\renewcommand*\insertshorttitle{%
  \oldmacro\hfill%
  \insertframenumber\,/\,\inserttotalframenumber}

\title[Enhanced Spatial Acuity via Concerted Firing]{Enhanced Spatial Acuity via Concerted Firing \\ of Retinal Ganglion Cells?}
\author[Daniel O'Shea (\textit{djoshea@stanford.edu})]{Daniel O'Shea \\
			\small{djoshea@stanford.edu} \\
 			\normalsize{\phantom{A}}}
\institute{\footnotesize{Baccus Lab Meeting}}
\date[]{\footnotesize{5 March 2010}}

\begin{document}
\textblockorigin{0cm}{0cm}

\frame{\titlepage}

\frame{\tableofcontents}

\section{Retinal Ganglion Population Supersampling}
\subsection{Purpose and Method}
\frame
{
  \frametitle{Retinal Ganglion Population Supersampling}

  \begin{block}{Low Density vs. High Density}
  	\begin{itemize}
		\item Low density array covers 700 $\mu$m$^2$ at 100 $\mu$m sampling
		\item High density array covers 150 $\mu$m$^2$ at 30 $\mu$m sampling
		\item Best of both worlds?
	\end{itemize}
  \end{block}
  \onslide<2->
  \begin{block}{Idea}
	\begin{itemize}
		\item Record with low density
		\item Pick up and translate retina relative to the array
		\item Set down and xrepeat
	\end{itemize}
  \end{block}
}

\subsection{Registration of Receptive Field Maps}
\frame
{
	\frametitle{Four Maps, Pre-Alignment}
	\begin{center}
	\includegraphics[width=90mm]{prealign.png}
	\end{center}
}

\frame
{
	\frametitle{Alignment Algorithm: \only<1>{Generate Candidate Offsets}\onslide<2>{Smooth, Pick Maximum}
	}
	\only<1>{
	\begin{center}
	\includegraphics[width=80mm]{fig_offsets.pdf}
	\end{center}}
	\onslide<2>
	\begin{center}
	\includegraphics[width=80mm]{fig_offsetsheat.pdf}
	\end{center}
}

\frame
{
	\frametitle{Four Maps: Post-Alignment}
	\begin{center}
	\includegraphics[width=90mm]{postalign.png}
	\end{center}
}

\frame
{
	\frametitle{Four Maps: Post-Merge}
	\begin{center}
	\includegraphics[width=90mm]{postmerge.png}
	\end{center}
}

\subsection{Pairwise Simultaneous Recording}
\frame
{
	\frametitle{Pairwise Simultaneity Matrix}
	\begin{center}
	\includegraphics[height=70mm]{pairwisesimul.png}
	\end{center}
}

\frame
{
	\frametitle{Adjacent Neighbor Simultaneity}
	\begin{center}
	\includegraphics[height=70mm]{adjacentsimul.png}
	\end{center}
}

\section{Event Triggered Analysis of Synchronous Firing}
\subsection{Synchronous Events in Retinal Ganglion Cells}
\frame
{
\begin{block}{Ubiquity of Synchronized Firing}
\begin{itemize}
	\item Found in RGC populations of goldfish, salamander, frog, rat, mouse, rabbit, guinea pig, cat, and macaque \\(Schlens et al. 2008)
	\item Dependency on functional type and receptive field separation; 500 $\mu$m spatial scale (Meister 1996)
	\end{itemize}
\end{block}
\onslide<2->
\begin{block}{Timescales (Mastronarde 1983)}
\begin{itemize}
	\item \textbf{50 ms}: Quantal fluctuations in shared photoreceptors
	\item \textbf{10 ms}: Shared presynaptic amacrine input
	\item \textbf{1 ms}: Gap junction coupling
\end{itemize}
\end{block}
}

\frame{
\begin{block}{Role in Retinal Signaling?}
\begin{itemize}
	\item<1-> Multiplexing in RGCs trade temporal bandwidth for spatial acuity (Meister  1996)
		\item<2-> Synchonization induced by fixational eye movements improves feature estimation, internal trigger signal \\(Greschner et al. 2002)
	\item<3-> Multineuronal codes convey higher resolution information than predicted from single cell analysis \\(Schnitzer and Meister 2003)
		\item<4-> Exploited in population decoding to recover 20\% more information (Pillow et al. 2008)
\end{itemize}
\end{block}
}

\subsection{Proportion of Synchronous Events in RF Map Data}
\frame{
\begin{block}{Synchronous Event Proportion}
\begin{itemize}
\item Consider only \textbf{adjacent spikes} across pair
\item If a pair of spikes (one from each cell) lie within $\Delta T$ ms, consider them a synchronous pair
\item Calculate the proportion of total spikes from both cells that are part of a synchronous pair
\end{itemize}
\end{block}

For a given pair of cells, measures the proportion of synchronous events sent out the optic nerve
}

\frame{
\frametitle{Synchronous Event Proportion: Distance and Cell Type}
$\Delta T$ = \only<1>{50}\only<2>{20}\only<3>{10}\only<4>{5}\only<5>{1} ms

\begin{center}
\only<1>{\includegraphics[height=60mm]{syncprop50.pdf}}
\only<2>{\includegraphics[height=60mm]{syncprop20.pdf}}
\only<3>{\includegraphics[height=60mm]{syncprop10.pdf}}
\only<4>{\includegraphics[height=60mm]{syncprop5.pdf}}
\only<5>{\includegraphics[height=60mm]{syncprop1.pdf}}
\end{center}

}

\subsection{Pairwise Relative Timing}
\frame
{
\begin{block}{Beyond Synchrony: Pairwise Relative Timing}
\begin{itemize}
\item RGCs encode the spatial structure of a briefly presented image in the relative timing of their first spikes (Gollisch and Meister 2008)
\end{itemize}
\end{block}
\onslide<2->
\begin{block}{What Information does Relative Timing Convey?}
\begin{itemize}
\item Extend LN models used for single cell analysis to generalized ``event times''
\item \textbf{Event triggered average} conveys best guess of stimulus preceding that event
\end{itemize}
\end{block}
}

\frame
{
\begin{block}{Extension of Synchronous Receptive Field}
\begin{itemize}
\item Receptive fields of synchronous events are smaller than both RFs and lie at the intersection (Meister 1995)
\end{itemize}
\end{block}

\begin{center}
\includegraphics[height=40mm]{meister1995.pdf}
\end{center}
}

\frame
{
\frametitle{Receptive Field Map (Single Recording)}

\begin{center}
\includegraphics[height=70mm]{rf5map.pdf}
\end{center}
}

\frame
{
\frametitle{\only<1>{Adjacent Spike Cross Correlation}\only<2>{Relative Timing RFs} (Adjacent, Wide)}

\only<1>{
\begin{center}
\includegraphics[height=70mm]{spikecorr1v2.pdf}
\end{center}
\begin{textblock*}{25mm}[0,0](82mm,22mm)%
\includegraphics[width=25mm]{rfmap1v2.pdf}
\end{textblock*}
}

\only<2>{
\begin{center}
\includegraphics[height=70mm]{syncpair1v2.pdf}
\end{center}
}
}

\frame
{
\frametitle{\only<1>{Adjacent Spike Cross Correlation}\only<2>{Relative Timing RFs} (Adjacent, Narrow)}

\only<1>{
\begin{center}
\includegraphics[height=70mm]{spikecorr2v33.pdf}
\end{center}
\begin{textblock*}{25mm}[0,0](82mm,22mm)%
\includegraphics[width=25mm]{rfmap2v33.pdf}
\end{textblock*}
}

\only<2>{
\begin{center}
\includegraphics[height=70mm]{syncpair2v33.pdf}
\end{center}
}
}

\frame
{
\frametitle{\only<1>{Adjacent Spike Cross Correlation}\only<2>{Relative Timing RFs} (Fast vs. Medium OFF)}

\only<1>{
\begin{center}
\includegraphics[height=70mm]{spikecorr20v21.pdf}
\end{center}
\begin{textblock*}{25mm}[0,0](82mm,22mm)%
\includegraphics[width=25mm]{rfmap20v21.pdf}
\end{textblock*}
}

\only<2>{
\begin{center}
\includegraphics[height=70mm]{syncpair20v21.pdf}
\end{center}
}
}

\frame{
\frametitle{Adjacent Pair Relative Timing RFs}

\begin{center}
\includegraphics[height=70mm]{syncpairadj.pdf}
\end{center}

}


\frame{
\frametitle{Relative Timing RFs \textbf{Span} Adjacent Pair RFs}

\begin{center}
\includegraphics[height=70mm]{syncrfspan.pdf}
\end{center}

}


\frame{
\frametitle{Relative Timing RFs \textbf{Interpolate} Between Adjacent Pairs}
\begin{center}
\includegraphics[height=70mm]{syncrflines.pdf}
\end{center}
}

\section{Conclusion}
\subsection{RGC Population Supersampling}
\frame{
\begin{block}{RGC Population Supersampling}
\begin{itemize}
\item Record, lift, translate process allows spatial supersampling of RGC population
\item Small shifts preserve simultaneity between pairs
\item Sampling bias persists, some cells are rarely sorted on low density array
\end{itemize}
\end{block}
}

\subsection{Synchronous and Relative Timing in Pairwise Activity}
\frame{
\begin{block}{Synchronous Events}
\begin{itemize}
\item Synchronous events are frequent among nearby pairs
\item Proportion of synchronous events decreases with distance, reduced across functional type
\end{itemize}
\end{block}

\begin{block}{Relative Timing RFs}
\begin{itemize}
\item Relative Timing RFs span the connecting line between adjacent cells
\item Smooth interpolation may allow for enhanced spatial acuity
\item Relative timing carries information at image onset; downstream circuitry could be utilized during fixation
\item Role of fixational eye movements?
\end{itemize}
\end{block}
}

\end{document}
